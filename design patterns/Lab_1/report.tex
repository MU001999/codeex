\documentclass{article}

\usepackage{color}
\usepackage{xcolor}
\definecolor{keywordcolor}{rgb}{0.8,0.1,0.5}
\usepackage{listings}
\lstset{breaklines}
\lstset{extendedchars=false}
\lstset{language=C++,
	keywordstyle=\color{keywordcolor} \bfseries,
	identifierstyle=,
	basicstyle=\ttfamily, 
	commentstyle=\color{blue} \textit,
	stringstyle=\ttfamily, 
	showstringspaces=false,
	%frame=shadowbox
	captionpos=b
}

\title{Report for Lab 1}
\author{Xu Bo}
\date{}

\begin{document}
	\maketitle
	
\section{Introduction}
    I choose the case 4 in lab 1. We take the stock market as the background in this lab. When a stock price changes by more than 10\%, the system sends a notification to investors who buy the stock. Each investor decides whether to continue holding, selling, or taking other operations based on the degree of rise or fall of the shares it holds. 
    
    In addition, the user can define the threshold of the price's rise and fall. The value, that is, does not exceed this threshold, the user does not want to be notified. Stocks can adjust their prices based on the number of purchases. In my design, if the number of shares sold exceeds 50 shares, the stock price will be raised. Otherwise, the stock price will be lowered.

\section{Key Components and Detailed Design}
    Obviously investors are the observers and stocks are the subjects, which will be observed by observers, in this case. So I designed four classes intuitively, two of these, named Subject and Observer, are abstract base classes and the other two, named Stock and Investor, are concrete classes.
    
\subsection{Subject}
    Subject is an abstract base class to declare interfaces but without implements. And subject will be observed by observers, while observers will be notified by subject when the subject updates. This is the class declaration:
    
\begin{lstlisting}
class Subject
{
  public:
    virtual ~Subject() = 0;
    virtual void registerObserver(Observer *observer) = 0;
    virtual void removeObserver(Observer *observer) = 0;
    virtual void notifyObservers() = 0;
};
\end{lstlisting}

    As we can see, there are three methods. Their functions are just like their names, method registerObserver is used to register the given observer in this subject, method removeObserver is used to remove the given observer in this subject and method notifyObservers is used to notify each observer that is registered in this subject.
    
\subsection{Observer}
    Class Observer is simple, it likes class Subject as an abstract base class. It provides only one pure virtual method, the interface, for all observers. This is the class declaration:
    
\begin{lstlisting}
class Observer
{
  public:
    virtual ~Observer() = 0;
    virtual void update(Subject *subject, double range) {};
};
\end{lstlisting}

    Actually this method is specific to class Investor, which will be mentioned later, because it has a parameter with type pointer of Subject and a parameter with type double. And we can see it has a default implement, what makes it easy to extend this system if there is another concrete observer to observe other subjects. For example, we can add a method likes:

\begin{lstlisting}
virtual void update(Subject *subject) {};
\end{lstlisting}

    This will not make errors when just add this declaration in this class because it has the default implement, so I think it's useful if we need to extend.

\subsection{Stock}
    Class Stock is the concrete class extends class Subject. It implements all methods inherited from Subject. And it contains a list of observers to store the registered observers and some specific information such as price of this stock and information related to each shareholder such as the number of shares purchased by each shareholder and the threshold. Of course it provides methods to get or set its member variables like name or price.
    
    Compared with the Subject, Stock provides more methods, this is the class declaration:

\begin{lstlisting}
class Stock final : public Subject
{
  public:
    Stock(std::string name, double price);

    std::string getName() const;
    double getPrice() const;
    void setPrice(double price);

    void registerObserver(Observer *observer) override;
    void removeObserver(Observer *observer) override;
    void notifyObservers() override;

    void changeShares(Observer *observer, int shares);
    void changeRange(Observer *observer, double range);

    void nextRound();

  private:
    constexpr static int SellThreshold = 50;
    constexpr static double DefaultRange = 0.2;

    void updatePrice();

    std::string name_;
    double price_;
    int curDealShares_;
    std::list<Observer *> observers_;
    std::map<Observer *, std::tuple<int, double, double>> infos_;
};
\end{lstlisting}

    As mentioned before, it has a member variable to store information related to each shareholder. The variable is a map, and its type is std::map<Observer *, std::tuple<int, double, double> >, first for shares, second for the average price of the investor bought and third for the threshold of degree of price's rise or fall in the tuple.
    
    So the first two methods are used to change the information of the given observer. And there is a variable named curDealShares to store changed shares in each round. So method changeShares won't update price but will change the curDealShares. And then we call method nextRound to update price when we are ready to go to next round. The following code is the implementation of method updatePrice:
    
\begin{lstlisting}
void Stock::updatePrice()
{
    // this code is copied from
    // https://en.cppreference.com/w/cpp/numeric/random/uniform_real_distribution
    // to produce a random floating-point value
    random_device rd;
    mt19937 gen(rd());
    uniform_real_distribution<> dis(0, 0.5);
    setPrice(price_ * (curDealShares_ <= SellThreshold
        ? (1 - dis(gen))
        : (1 + dis(gen))));
}
\end{lstlisting}

    As we can see, \textbf{the degree of rise or fall is random}.

\subsection{Investor}

    Class Investor is the concrete class extends class Observer. So it observes some subjects, some stocks in fact, and waits for their notifications. This class not only implements the method update inherited from Observer but also has other 6 methods:
    
\begin{lstlisting}
class Investor final : public Observer
{
  public:
    using ActionType = std::function<void(Investor *, Stock *, double)>;

    Investor(std::string name, ActionType &&action = nullptr);

    std::string getName() const;
    void setUpdateAction(ActionType &&action);
    
    void update(Subject* stock, double range) override;
    void buyStock(Stock *stock, int shares, double range = -1.0);
    void changeRange(Stock *stock, double range);
    void sellStock(Stock *stock, int shares);
    void sellStockAll(Stock *stock);

  private:
    std::string name_;
    ActionType updateAction_;
    std::list<Stock *> stocks_;
    std::map<Stock *, int> shares_;
};
\end{lstlisting}

    I think it's clear to see methods named getName and setUpdateAction are just getter and setter methods for name\_ and updateAction\_. In deed, \textbf{what is actually executed in the method update, inherited from Observer, is the function object updateAction\_}. So what we can see in method update is:
    
\begin{lstlisting}
void Investor::update(Subject *stock, double range)
{
    updateAction_(this, dynamic_cast<Stock *>(stock), range);
}
\end{lstlisting}
    
    Method changeRange is simple to call the corresponding stock's method changeRange to change the threshold of degree of price's rise or fall to decide to notify or not.
    
    And there are three methods left, buyStock, sellStock and sellStockAll. Method buyStock will register the investor in the given stock if not registered, buy the given shares of the given stock and set range in stock if range is greater than zero. Method sellStock is used to sell the given shares of the given stock and remove the stock from stocks\_ and shares\_ after selling if the given shares is greater than shares the investor has. Method sellStockAll is simple to call sellStock.

\section{Test Process}

    In the previous section, we have already known some about these classes. So let's start to see how the entire system works now!
    
    Actually the market is running in rounds and we can treat transactions made in the same round as transactions that are made at the same time. \textbf{That is, each round is equivalent to every moment}, but we can control the rounds manually by calling stocks' method nextRound.
    
\subsection{Firstly}    
    
    Initially we can declare a few stocks, or only one stock, and a few investors. The investors' actions for updating are set by user. In this test, action of investor is that the investor will purchase 100 shares of the stock if price rises, or sell 20 shares of the stock if price falls. I declare one stock and three investors:
    
\begin{lstlisting}
void simpleAction(Investor *investor, Stock *stock, double range)
{
    if (range < 0)
    {
        investor->buyStock(stock, 100);
    }
    else
    {
        investor->sellStock(stock, 20);
    }
}

Stock Tfosorcim("Tfosorcim", 2000);

Investor XiaoMin("XiaoMin", simpleAction),
         DaXiong("DaXiong", simpleAction),
         Michael("Michael", simpleAction);

\end{lstlisting}

    In fact, there are some helpful strings will be printed when calling simpleAction but not in above code.
    
\subsection{Secondly}    
  
    Now investors can buy or not buy shares of stocks they like. The investors will be registered in stocks which they not purchased previously. In this test, the three investors only buy stocks and set their ranges in the first round:
    
\begin{lstlisting}
XiaoMin.buyStock(&Tfosorcim, 80, 0.1);
DaXiong.buyStock(&Tfosorcim, 60, 0.2);
Michael.buyStock(&Tfosorcim, 40, 0.3);
\end{lstlisting}
    
    The stocks will also record how many deals happen in this round and count the account of changed shares and decide to update their prices by some ways related to the changed shares or just random ways.
    
    We can call stocks' method nextRound if no deals will happen in this round now. In my design, if the shares sold are more than 50 shares, stocks' prices will rise randomly, otherwise the prices will fall randomly as mentioned above.
    
    At present, stocks' prices have been updated, and stocks will notify investors who purchase shares of their if the degrees of prices change meet the investors' conditions. \textbf{As we can see, the operation to decide whether to notify is in stocks, although the conditions are provided by investors}.
    
    Now method 'update' of investors will be called by stocks, then the function object updateAction\_ of investors will be called later.
    
    Before we end each round, investors still can purchase or sell shares. These behaviors also won't change the stocks' prices immediately. Naturally the last step in each round is calling stocks' method nextRound. After that we only need to repeat these steps, the market will always run, and can stop at any time.
    
    In this test, I run 10 rounds and not do other things except in first round. I just call stock's method nextRound in each round as the following code:
    
\begin{lstlisting}
void round(int num)
{
    cout << "This is Round " << num << "!" << endl;
    Tfosorcim.nextRound();
    showStock(); cout << endl;
}
\end{lstlisting}

\subsection{Result}

    Because the action in stock's method updatePrice is random, so the result can't be reproduced. This is my last execution's output:
    
\begin{lstlisting}
Tfosorcim's price is 2000
Then the stock market is starting to work
XiaoMin buys 80 shares of Tfosorcim
DaXiong buys 60 shares of Tfosorcim
Michael buys 40 shares of Tfosorcim

This is Round 1!
        My name is XiaoMin. The stock of Tfosorcim is up 48%. I'm goint to sell 20 shares because I don't like to earn much money although I believe that it will keep going up :)
        My name is DaXiong. The stock of Tfosorcim is up 48%. I'm goint to sell 20 shares because I don't like to earn much money although I believe that it will keep going up :)
        My name is Michael. The stock of Tfosorcim is up 48%. I'm goint to sell 20 shares because I don't like to earn much money although I believe that it will keep going up :)
Tfosorcim's price is 2961.01

This is Round 2!
        My name is XiaoMin. The stock of Tfosorcim is up 35%. I'm goint to sell 20 shares because I don't like to earn much money although I believe that it will keep going up :)
        My name is DaXiong. The stock of Tfosorcim is up 49%. I'm goint to sell 20 shares because I don't like to earn much money although I believe that it will keep going up :)
        My name is Michael. The stock of Tfosorcim is up 119%. I'm goint to sell 20 shares because I don't like to earn much money although I believe that it will keep going up :)
Tfosorcim's price is 2277.47

This is Round 3!
        My name is XiaoMin. The stock of Tfosorcim is up 17%. I'm goint to sell 20 shares because I don't like to earn much money although I believe that it will keep going up :)
        My name is DaXiong. The stock of Tfosorcim is up 113%. I'm goint to sell 20 shares because I don't like to earn much money although I believe that it will keep going up :)
Tfosorcim's price is 1626.69

This is Round 4!
        My name is XiaoMin. The stock of Tfosorcim fell 17%. I'm going to buy more 100 shares, or rather I've been trapped :(
Tfosorcim's price is 936.314

This is Round 5!
Tfosorcim's price is 950.936

This is Round 6!
Tfosorcim's price is 908.973

This is Round 7!
        My name is XiaoMin. The stock of Tfosorcim fell 30%. I'm going to buy more 100 shares, or rather I've been trapped :(
Tfosorcim's price is 677.162

This is Round 8!
Tfosorcim's price is 852.582

This is Round 9!
        My name is XiaoMin. The stock of Tfosorcim fell 19%. I'm going to buy more 100 shares, or rather I've been trapped :(
Tfosorcim's price is 675.572

This is Round 10!
        My name is XiaoMin. The stock of Tfosorcim is up 23%. I'm goint to sell 20 shares because I don't like to earn much money although I believe that it will keep going up :)
Tfosorcim's price is 968.78
\end{lstlisting}

\section{Conclusion}

    Observer pattern is to define a one-to-many dependency between objects so that when one object changes state, all its dependents are notified and updated automatically.
    
    I think the more difficult part is to think about the relationship between stock price updates and the actions of investors. 
    
    And I think I have learned the basic elements of this pattern.

\end{document}
